\begin{abstract}
Research in $k$-best computation and enumeration has historically been limited almost exclusively to problems with known polynomial-time solutions. We expand the field of study by focusing on the computational complexity of finding the second-best and $k$-best solutions to $\nph{}$ combinatorial optimization problems. We introduce several new problem classes and analyze their complexity for a wide range of well-known combinitorial optimization problems. Our results show that, in most cases, finding the second-best or $k$-best solution is itself an $NP$-hard task, even when the optimal solution is provided as part of the input. We also develop a general framework for proving the hardness of these problems in the future as well as identifying  notable exceptions to the general rule. This thesis lays the foundation for future research on the complexity of $k$-best optimization problems and highlights the potential for further investigations into $k$ and second-best $\nph{}$ problems.
\end{abstract}


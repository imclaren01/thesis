\chapter{Future Research}
Despite what feels like quite a number of results in the sections above, there is clearly a considerable amount of space for future work to be done in the field. 
Examining the Traveling Salesman Problem, as we mentioned, we have not yet formalized the proof that \exob{}-Euclidean-TSP and \inob{}-Euclidean-TSP are $\nph{}$. Future work should aim to fill this gap. There are also many natural restrictions to TSP other than the simple convex shape constraint that can also make the optimization problem computable in polynomial time. For instance, Rothe published a result that graphs constrained to be a small number of parallel lines or a ``necklace tour" permit polynomial time solutions to the traveling salesman problem \cite{rothe1988two}. We would expect future work on the second-best question to explore whether these constraints also permit easy solutions to the second-best versions of TSP.

Beyond the traveling salesman, there are many natural combinitorial optimization problems to be examined. In particular, suspect it may be easier to find second-best solutions to optimization problems that permit approximations of arbitrary accuracy, largely considered to be easier problems to compute than other problems in $\nph{}$. Some of examples of these problems are Maximum Planar Independent Set and Knapsack. We would also look for future work to expand our results to the remaining problems in Karp's 21 $\npc{}$ problems defined by in \cite{karp1972reducibility}.

Finally, we would of course like to encourage further work in developing efficient algorithms to calculate second-best solutions and enumerate $k$-best solutions for known polynomial time problems. For instance, further exploration into the details behind the Minimum Spanning Tree problem being easy to enumerate for a constant $k$ but being $\nph{}$ for a variable $k$ could be an interesting area of research. Other such problems that could use further exploration is that of $k$-best shortest paths and maximum matchings.
